\documentclass[11pt]{article}

\newcommand{\numpy}{{\tt numpy}}    % tt font for numpy

\usepackage{enumitem}

\topmargin -.5in
\textheight 9in
\oddsidemargin -.25in
\evensidemargin -.25in
\textwidth 7in
\date{}

\begin{document}

% ========== Edit your name here
\author{Pedro Martins Pereira - 2124290019}
\title{Trabalho 01 - Lógica de Programação}
\maketitle

\medskip

\begin{enumerate}

\item
	

	Resposta:

	\begin{enumerate}[label=(\alph*)]
		\item \addtocounter{enumii}{0}
		\item 34857 é um literal do tipo number.
		\item "true" é um lietral do tipo string.
		\item false é do tipo boolean.
		\item "304958" é um literal do tipo string.
		\item -2343 é um literal do tipo number.
		\item "23/12/99" é um literal do tipo string. 		  \item true é um lietral do tipo boolean.
		\item "NOME" é um lietral do tipo string
		\item 0.5 é um literal do tipo number.
		\item NOME, por não estar entre aspas não está definido e causará o programa a ser interrompido com uma excessão caso seja executado. 
		\item 'i' é um literal do tipo string. 
  	\end{enumerate}

\item Resposta:

	\begin{enumerate}[label=(\alph*)]
		\item \addtocounter{enumii}{0}
		\item string, já que podem ocorrer zeros à direita.
		\item string, já que podem ocorrer zeros à direita. 
		\item boolean.
		\item number.
		\item string.
  	\end{enumerate}


\item Resposta:

	\begin{enumerate}[label=(\alph*)]
  		\item Não, já que um placa de automóvel é composta por letras e números. Além disso placas de automóveis podem ter zeros à esquerda, os quais são ignorados em tipos numéricos.
  		\item Falso, variáveis do tipo string podem conter quaisquer caracteres da tabela ascii, numerais e até mesmo caracteres unicode.
  		\item Verdadeiro. true e false são os dois valores booleanos disponíveis em JavaScript e representam as constantes lógicas da lógica clássica ou álgebra de boole. 
  		\item 
	\end{enumerate}

\item 
	Prove a validade do argumento seguinte usando regras de inferência (se necessário utilize também as de equivalência).
	\begin{itemize}
		\item Gabriel estuda ou não está cansado.
		\item Se Gabriel estuda, então dorme tarde.
		\item Gabriel não dorme tarde ou está cansado. 
	\end{itemize}
	Portanto, Gabriel está cansado se, e somente se estuda (conclusão).

	Resposta:

	Considere que "Gabriel estuda" seja representado por $\varphi$ e "Gabriel está cansado" por $\psi$. Considere também que "Gabriel dorme tarde" seja $\chi$.   

	Então o argumento tem a seguinte forma:
	\begin{enumerate}[label=\roman*]
  		\item $\varphi \vee \neg \psi$ 
  		\item $\varphi \rightarrow \chi$
  		\item $\neg \chi \vee \psi$ 
		\item $\models \psi \leftrightarrow \varphi$ 
	\end{enumerate}

	Faltam passos que determiem a conclusão, transformando em uma dedução correta:

	\begin{enumerate}[label=\roman*]
  		\item $\varphi \rightarrow \chi$ \textit{premissa}
  		\item $\neg \chi \vee \psi$ \textit{premissa}
  		\item $\chi \rightarrow \psi $ \textit{De Morgan}
  		\item $\varphi \rightarrow  \psi$ \textit{transitividade}
  		\item $\neg \psi \rightarrow  \neg \varphi$ \textit{contrapositiva}
  		\item $\neg\neg \psi \rightarrow  \neg\neg \varphi$ \textit{inversa}
  		\item $\psi \rightarrow  \varphi$ \textit{eliminação da dupla negação}
		\item $\models \psi \leftrightarrow \varphi$  \textit{conjunção iv, vii} 
	\end{enumerate}

\item Utilize as regras de equivalência e encontre a opção correta. \smallskip 
	Dizer que: Pedro não é pedreiro ou Paulo é paulista. É do ponto de vista lógico, o mesmo que dizer que:\smallskip

	Resposta:
	Se Pedro é pedreiro, então Paulo é paulista.

\end{enumerate}

\end{document}
\grid
\grid



\documentclass{scrreprt}
\makeatletter
\usepackage{color}
\definecolor{lightgray}{rgb}{0.95, 0.95, 0.95}
\definecolor{darkgray}{rgb}{0.4, 0.4, 0.4}
%\definecolor{purple}{rgb}{0.65, 0.12, 0.82}
\definecolor{editorGray}{rgb}{0.95, 0.95, 0.95}
\definecolor{editorOcher}{rgb}{1, 0.5, 0} % #FF7F00 -> rgb(239, 169, 0)
\definecolor{editorGreen}{rgb}{0, 0.5, 0} % #007C00 -> rgb(0, 124, 0)
\definecolor{orange}{rgb}{1,0.45,0.13}		
\definecolor{olive}{rgb}{0.17,0.59,0.20}
\definecolor{brown}{rgb}{0.69,0.31,0.31}
\definecolor{purple}{rgb}{0.38,0.18,0.81}
\definecolor{lightblue}{rgb}{0.1,0.57,0.7}
\definecolor{lightred}{rgb}{1,0.4,0.5}
\usepackage{upquote}
\usepackage{listings}
% CSS
\lstdefinelanguage{CSS}{
  keywords={color,background-image:,margin,padding,font,weight,display,position,top,left,right,bottom,list,style,border,size,white,space,min,width, transition:, transform:, transition-property, transition-duration, transition-timing-function},	
  sensitive=true,
  morecomment=[l]{//},
  morecomment=[s]{/*}{*/},
  morestring=[b]',
  morestring=[b]",
  alsoletter={:},
  alsodigit={-}
}

% JavaScript
\lstdefinelanguage{JavaScript}{
  morekeywords={typeof, new, true, false, catch, function, return, null, catch, switch, var, if, in, while, do, else, case, break},
  morecomment=[s]{/*}{*/},
  morecomment=[l]//,
  morestring=[b]",
  morestring=[b]'
}

\lstdefinelanguage{HTML5}{
  language=html,
  sensitive=true,	
  alsoletter={<>=-},	
  morecomment=[s]{<!-}{-->},
  tag=[s],
  otherkeywords={
  % General
  >,
  % Standard tags
	<!DOCTYPE,
  </html, <html, <head, <title, </title, <style, </style, <link, </head, <meta, />,
	% body
	</body, <body,
	% Divs
	</div, <div, </div>, 
	% Paragraphs
	</p, <p, </p>,
	% scripts
	</script, <script,
  % More tags...
  <canvas, /canvas>, <svg, <rect, <animateTransform, </rect>, </svg>, <video, <source, <iframe, </iframe>, </video>, <image, </image>, <header, </header, <article, </article
  },
  ndkeywords={
  % General
  =,
  % HTML attributes
  charset=, src=, id=, width=, height=, style=, type=, rel=, href=,
  % SVG attributes
  fill=, attributeName=, begin=, dur=, from=, to=, poster=, controls=, x=, y=, repeatCount=, xlink:href=,
  % properties
  margin:, padding:, background-image:, border:, top:, left:, position:, width:, height:, margin-top:, margin-bottom:, font-size:, line-height:,
	% CSS3 properties
  transform:, -moz-transform:, -webkit-transform:,
  animation:, -webkit-animation:,
  transition:,  transition-duration:, transition-property:, transition-timing-function:,
  }
}

\lstdefinestyle{htmlcssjs} {%
  % General design
%  backgroundcolor=\color{editorGray},
  basicstyle={\footnotesize\ttfamily},   
  frame=b,
  % line-numbers
  xleftmargin={0.75cm},
  numbers=left,
  stepnumber=1,
  firstnumber=1,
  numberfirstline=true,	
  % Code design
  identifierstyle=\color{black},
  keywordstyle=\color{blue}\bfseries,
  ndkeywordstyle=\color{editorGreen}\bfseries,
  stringstyle=\color{editorOcher}\ttfamily,
  commentstyle=\color{brown}\ttfamily,
  % Code
  language=HTML5,
  alsolanguage=JavaScript,
  alsodigit={.:;},	
  tabsize=2,
  showtabs=false,
  showspaces=false,
  showstringspaces=false,
  extendedchars=true,
  breaklines=true,
  % German umlauts
  literate=%
  {Ö}{{\"O}}1
  {Ä}{{\"A}}1
  {Ü}{{\"U}}1
  {ß}{{\ss}}1
  {ü}{{\"u}}1
  {ä}{{\"a}}1
  {ö}{{\"o}}1
}
%
\lstdefinestyle{py} {%
language=python,
literate=%
*{0}{{{\color{lightred}0}}}1
{1}{{{\color{lightred}1}}}1
{2}{{{\color{lightred}2}}}1
{3}{{{\color{lightred}3}}}1
{4}{{{\color{lightred}4}}}1
{5}{{{\color{lightred}5}}}1
{6}{{{\color{lightred}6}}}1
{7}{{{\color{lightred}7}}}1
{8}{{{\color{lightred}8}}}1
{9}{{{\color{lightred}9}}}1,
basicstyle=\footnotesize\ttfamily, % Standardschrift
numbers=left,               % Ort der Zeilennummern
%numberstyle=\tiny,          % Stil der Zeilennummern
%stepnumber=2,               % Abstand zwischen den Zeilennummern
numbersep=5pt,              % Abstand der Nummern zum Text
tabsize=4,                  % Groesse von Tabs
extendedchars=true,         %
breaklines=true,            % Zeilen werden Umgebrochen
keywordstyle=\color{blue}\bfseries,
frame=b,
commentstyle=\color{brown}\itshape,
stringstyle=\color{editorOcher}\ttfamily, % Farbe der String
showspaces=false,           % Leerzeichen anzeigen ?
showtabs=false,             % Tabs anzeigen ?
xleftmargin=17pt,
framexleftmargin=17pt,
framexrightmargin=5pt,
framexbottommargin=4pt,
%backgroundcolor=\color{lightgray},
showstringspaces=false,      % Leerzeichen in Strings anzeigen ?
}%
%
\makeatother
\begin{document}
\begin{lstlisting}[style=htmlcssjs]
/******************** EXERCICIO 5 **************************/
/* a. */console.log(Math.sqrt((45*2-30/5-8),2.0)) // 8.717797887081348
/* b. */console.log(Math.pow(-3, 3)) // -27 
/* c. */console.log(27 - Math.pow(3, 4)) // -54 
/* d. */console.log(70 - (50/2)*5*3) // -305
/* e. */console.log(Math.round(Math.pow(-70.75, 4))) // 25055656 

/******************** EXERCICIO 6 **************************/
let A,B,C,D,E

A = 20 // => 20

B = (40 + A)/3 
/* => (20 + 20) -> 70 
 * 70/3 -> 20 */

C = Math.sqrt((A + 80), 2.0)
/* => 20 + 80 -> 100
 * => sqrt(100,2.0) -> 10
 * */

D = (A >= B) // => (20 >= 20) -> true

E = (C == B) // => (10 == 20) -> false

console.log(A, " ", " ", B, " ", C, " ", D, " ", E)


/******************** EXERCICIO 7 **************************/

/* a. */ console.log("mario" == "maria") // false
/* b. */ console.log(2 + 4 == 6) // true
/* c. */ console.log(10 - 4 > 7) // false
/* d. */ console.log((2*3)>(3*2)) // false
/* e. */ console.log(!('a' > 'A')) // ????

/* Resultado: false, true, false, false, false */


/******************** EXERCICIO 8 **************************/

function retornaMaiorNumero(n1, n2, n3) {
	return Math.max(n1, n2, n3)
}

const input1 = require('prompt-sync')({sigint:true})
const input2 = require('prompt-sync')({sigint:true})
const input3 = require('prompt-sync')({sigint:true})

const num1 = input1("Insira o primeiro numero ")
const num2 = input2("Insira o segundo numero ")
const num3 = input3("Insira o terceiro numero ")

console.log(num1, num2, num3)

console.log("O maior numero e ", retornaMaiorNumero(num1, num2, num3))



/******************** EXERCICIO 9 **************************/

let myArray = [3, 12, 4, 15, 1 , 2, 7, 8]

function imprimeDoisMaiores(array) { 
	var max = Math.max.apply(null, array) /* Encontrando o maior elemento */ 
	console.log("Maior elemento:", max)

	array.splice(array.indexOf(max), 1) /* Excluindo o maior elemento do array */ 

	console.log("Segundo maior elemento:", Math.max.apply(null, array)) /* Encontrando e imprimindo o proximo maior */ 
}

imprimeDoisMaiores(myArray)


/******************** EXERCICIO 10 **************************/

const ateFicarMaior = (altura1, taxaDeCrescimento1, altura2, taxaDeCrescimento2) => {
	let ano = 0
	for (ano = 0; altura2 <= altura1; ano++) {
		altura1 += taxaDeCrescimento1
		altura2 += taxaDeCrescimento2
	}
	return ano
}

let anacleto = {
	alturaCM: 150,
	taxaDeCrescimentoAno: 2 
}

let felisberto = {
	alturaCM: 110,
	taxaDeCrescimentoAno: 3 
}


let passou = ateFicarMaior(anacleto.alturaCM, anacleto.taxaDeCrescimentoAno, felisberto.alturaCM, felisberto.taxaDeCrescimentoAno)

console.log("Anacleto passara Felisberto apos " + passou + " anos, estando com 2m33cm e seu irmao com 2m32cm!")

\end{lstlisting}
\end{document}

\documentclass[11pt]{article}

\newcommand{\numpy}{{\tt numpy}}    % tt font for numpy

\usepackage{enumitem}

\usepackage{listings} %code highlighter
\usepackage{color} %use color
\definecolor{mygreen}{rgb}{0,0.6,0}
\definecolor{mygray}{rgb}{0.5,0.5,0.5}
\definecolor{mymauve}{rgb}{0.58,0,0.82}
 
%Customize a bit the look
\lstset{ %
backgroundcolor=\color{white}, % choose the background color; you must add \usepackage{color} or \usepackage{xcolor}
basicstyle=\footnotesize, % the size of the fonts that are used for the code
breakatwhitespace=false, % sets if automatic breaks should only happen at whitespace
breaklines=true, % sets automatic line breaking
captionpos=b, % sets the caption-position to bottom
commentstyle=\color{mygreen}, % comment style
deletekeywords={...}, % if you want to delete keywords from the given language
escapeinside={\%*}{*)}, % if you want to add LaTeX within your code
extendedchars=true, % lets you use non-ASCII characters; for 8-bits encodings only, does not work with UTF-8
frame=single, % adds a frame around the code
keepspaces=true, % keeps spaces in text, useful for keeping indentation of code (possibly needs columns=flexible)
keywordstyle=\color{blue}, % keyword style
% language=Octave, % the language of the code
morekeywords={*,...}, % if you want to add more keywords to the set
numbers=left, % where to put the line-numbers; possible values are (none, left, right)
numbersep=5pt, % how far the line-numbers are from the code
numberstyle=\tiny\color{mygray}, % the style that is used for the line-numbers
rulecolor=\color{black}, % if not set, the frame-color may be changed on line-breaks within not-black text (e.g. comments (green here))
showspaces=false, % show spaces everywhere adding particular underscores; it overrides 'showstringspaces'
showstringspaces=false, % underline spaces within strings only
showtabs=false, % show tabs within strings adding particular underscores
stepnumber=1, % the step between two line-numbers. If it's 1, each line will be numbered
stringstyle=\color{mymauve}, % string literal style
tabsize=2, % sets default tabsize to 2 spaces
title=\lstname % show the filename of files included with \lstinputlisting; also try caption instead of title
}
%END of listing package%
 
\definecolor{darkgray}{rgb}{.4,.4,.4}
\definecolor{purple}{rgb}{0.35, 0.3, 0.82}
 
%define Javascript language
\lstdefinelanguage{JavaScript}{
keywords={typeof, new, true, false, catch, function, return, null, catch, switch, var, if, in, while, do, else, case, break},
keywordstyle=\color{blue}\bfseries,
ndkeywords={class, export, boolean, throw, implements, import, this},
ndkeywordstyle=\color{darkgray}\bfseries,
identifierstyle=\color{black},
sensitive=false,
comment=[l]{//},
morecomment=[s]{/*}{*/},
commentstyle=\color{purple}\ttfamily,
stringstyle=\color{red}\ttfamily,
morestring=[b]',
morestring=[b]"
}
 
\lstset{
language=JavaScript,
extendedchars=true,
basicstyle=\footnotesize\ttfamily,
showstringspaces=false,
showspaces=false,
numbers=left,
numberstyle=\footnotesize,
numbersep=9pt,
tabsize=2,
breaklines=true,
showtabs=false,
captionpos=b
}


\topmargin -.5in
\textheight 9in
\oddsidemargin -.25in
\evensidemargin -.25in
\textwidth 7in
\date{}

\begin{document}

% ========== Edit your name here
\author{Pedro Martins Pereira - 2124290019}
\title{Trabalho 01 - Lógica de Programação}
\maketitle

\medskip

\begin{enumerate}

\item
	

	Resposta:

	\begin{enumerate}[label=(\alph*)]
		\item \addtocounter{enumii}{0}
		\item 34857 é um literal do tipo number.
		\item "true" é um lietral do tipo string.
		\item false é do tipo boolean.
		\item "304958" é um literal do tipo string.
		\item -2343 é um literal do tipo number.
		\item "23/12/99" é um literal do tipo string. 		  \item true é um lietral do tipo boolean.
		\item "NOME" é um lietral do tipo string
		\item 0.5 é um literal do tipo number.
		\item NOME, por não estar entre aspas não está definido e causará o programa a ser interrompido com uma excessão caso seja executado. 
		\item 'i' é um literal do tipo string. 
  	\end{enumerate}

\item Resposta:

	\begin{enumerate}[label=(\alph*)]
		\item \addtocounter{enumii}{0}
		\item string, já que podem ocorrer zeros à esquerda.
		\item string, já que podem ocorrer zeros à esquerda. 
		\item boolean.
		\item number.
		\item string.
  	\end{enumerate}


\item Resposta:

	\begin{enumerate}[label=(\alph*)]
  		\item Não, já que um placa de automóvel é composta por letras e números. Além disso placas de automóveis podem ter zeros à esquerda, os quais são ignorados em tipos numéricos.
  		\item Falso, variáveis do tipo string podem conter quaisquer caracteres da tabela ascii, numerais e até mesmo caracteres unicode.
  		\item Verdadeiro. true e false são os dois valores booleanos disponíveis em JavaScript e representam as constantes lógicas da lógica clássica ou álgebra de boole. 
  		\item 
	\end{enumerate}

\item Resposta: 

(b)
\begin{lstlisting}[language=JavaScript]
let carro = {
	placa: "JDR0312",
	marca: "Toyota",
	modelo: "Supra" ,
	ano: 1995,
	precoFIPE: 32_735.00,
	disponivel: true
}
\end{lstlisting}

(c)
\begin{lstlisting}[language=JavaScript]
let  musica = {
	titulo:  "Portrait of Tracy",
	artistas: ['Jaco Pastorius'],
	generos: ['Jazz', 'Funk', 'Soul'],
	duracaoMinutos: 2.23
}

let artista = {
	nome: "Jaco Pastorius",
	musicas: ['Donna Lee', 'Continuum', 'Opus Pocus', 'Punk Jazz']
	generos: ['Jazz', 'Funk', 'Soul']
}

let colecao = {
	artistas = ['Jaco Pastorius', 'Victor Wooten', 'Ray Brown', 'Charles Mingus'],
	numeroDeAlbums = 45
}
\end{lstlisting}


(d)
\begin{lstlisting}[language=JavaScript]
let disciplina = {
	nome: "Logica de Programacao"
	professor: "Michel",
	alunos: [Pedro, Joao, Maria, Gustavo]
	mediaMinima: 5
} 

let coordenador = "Marcelo"
let linguagensDeProgramacao = ['JavaScript', 'HTML']
\end{lstlisting}

\newpage

(e)
\begin{lstlisting}[language=JavaScript]
let carrinhoDeCompras = ['Thinkpad', 'Camiseta Vermelha', 'Switch de Rede']
let cliente = {
	nome: "Richard Stallman",
	carrinho: carrinhoDeCompras, 
	cpf: "486.842.680-08"
}
\end{lstlisting}



\end{enumerate}
\end{document}
\grid
\grid



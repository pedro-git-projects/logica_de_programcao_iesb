\documentclass[11pt]{article}

\newcommand{\numpy}{{\tt numpy}}    % tt font for numpy

\usepackage{enumitem}

\topmargin -.5in
\textheight 9in
\oddsidemargin -.25in
\evensidemargin -.25in
\textwidth 7in
\date{}

\begin{document}

% ========== Edit your name here
\author{Pedro Martins Pereira - 2124290019}
\title{ADS 1A: Fundamentos de Lógica, Fórum 1}
\maketitle

\medskip

\begin{enumerate}

\item
	Sabendo que o valor lógico da proposição $p$ é verdadeiro e que o valor lógico da proposição $q$ é falsa, determine o valor lógico da seguinte fórmula $P$:

	$P: \neg(p \vee q) \leftrightarrow p \wedge q$

	Resposta:

	\begin{enumerate}[label=(\roman*)]
  		\item $p \vee q$ é verdadeiro, portanto
  		\item $\neg(p \vee q)$ é falso
  		\item $p\wedge q$ é falso, logo 
		\item $P: \neg(p \vee q) \leftrightarrow p \wedge q$ é verdadeiro
	\end{enumerate}

	Note, entretanto, que esta fórmula é verdadeira apenas para essa interpretação, ou linha da tabela verdade, não sendo válida, isto é verdadeira em todas interpretações.

\item
	Classifique as fórmulas abaixo como tatuologias, satisfatívies ou contraditórias:
	\begin{enumerate}[label=(\alph*)]
  		\item $(p \rightarrow q) \wedge (q \rightarrow r) \rightarrow (p \rightarrow r)$ 
  		\item $((p \wedge q)\rightarrow r) \leftrightarrow ((p \wedge \neg r)\rightarrow \neg q)$
  		\item $(p \rightarrow(q \rightarrow r))\rightarrow ((p \wedge q)\rightarrow r)$
  		\item $(p \rightarrow q) \leftrightarrow (\neg p \rightarrow \neg q)$
	\end{enumerate}

	Resposta:

	\begin{enumerate}[label=(\alph*)]
  		\item Tautologia 
  		\item Tautologia
  		\item Tautologia 
  		\item Satisftatível
	\end{enumerate}

\item 
	Prove a validade do argumento seguinte usando regras de inferência (se necessário utilize também as de equivalência).
	\begin{itemize}
		\item Gabriel estuda ou não está cansado.
		\item Se Gabriel estuda, então dorme tarde.
		\item Gabriel não dorme tarde ou está cansado. 
	\end{itemize}
	Portanto, Gabriel está cansado se, e somente se estuda (conclusão).

	Resposta:

	Considere que "Gabriel estuda" seja representado por $\varphi$ e "Gabriel está cansado" por $\psi$. Considere também que "Gabriel dorme tarde" seja $\chi$.   

	Então o argumento tem a seguinte forma:
	\begin{enumerate}[label=\roman*]
  		\item $\varphi \vee \neg \psi$ 
  		\item $\varphi \rightarrow \chi$
  		\item $\neg \chi \vee \psi$ 
		\item $\models \psi \leftrightarrow \varphi$ 
	\end{enumerate}

	Faltam passos que determiem a conclusão, transformando em uma dedução correta:

	\begin{enumerate}[label=\roman*]
  		\item $\varphi \rightarrow \chi$ \textit{premissa}
  		\item $\neg \chi \vee \psi$ \textit{premissa}
  		\item $\chi \rightarrow \psi $ \textit{De Morgan}
  		\item $\varphi \rightarrow  \psi$ \textit{transitividade}
  		\item $\neg \psi \rightarrow  \neg \varphi$ \textit{contrapositiva}
  		\item $\neg\neg \psi \rightarrow  \neg\neg \varphi$ \textit{inversa}
  		\item $\psi \rightarrow  \varphi$ \textit{eliminação da dupla negação}
		\item $\models \psi \leftrightarrow \varphi$  \textit{conjunção iv, vii} 
	\end{enumerate}

\item Utilize as regras de equivalência e encontre a opção correta. \smallskip 
	Dizer que: Pedro não é pedreiro ou Paulo é paulista. É do ponto de vista lógico, o mesmo que dizer que:\smallskip

	Resposta:
	Se Pedro é pedreiro, então Paulo é paulista.

\end{enumerate}

\end{document}
\grid
\grid



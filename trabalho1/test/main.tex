\documentclass{scrreprt}
\makeatother
%Define the listing package
\usepackage{listings} %code highlighter
\usepackage{color} %use color
\definecolor{mygreen}{rgb}{0,0.6,0}
\definecolor{mygray}{rgb}{0.5,0.5,0.5}
\definecolor{mymauve}{rgb}{0.58,0,0.82}
 
%Customize a bit the look
\lstset{ %
backgroundcolor=\color{white}, % choose the background color; you must add \usepackage{color} or \usepackage{xcolor}
basicstyle=\footnotesize, % the size of the fonts that are used for the code
breakatwhitespace=false, % sets if automatic breaks should only happen at whitespace
breaklines=true, % sets automatic line breaking
captionpos=b, % sets the caption-position to bottom
commentstyle=\color{mygreen}, % comment style
deletekeywords={...}, % if you want to delete keywords from the given language
escapeinside={\%*}{*)}, % if you want to add LaTeX within your code
extendedchars=true, % lets you use non-ASCII characters; for 8-bits encodings only, does not work with UTF-8
frame=single, % adds a frame around the code
keepspaces=true, % keeps spaces in text, useful for keeping indentation of code (possibly needs columns=flexible)
keywordstyle=\color{blue}, % keyword style
% language=Octave, % the language of the code
morekeywords={*,...}, % if you want to add more keywords to the set
numbers=left, % where to put the line-numbers; possible values are (none, left, right)
numbersep=5pt, % how far the line-numbers are from the code
numberstyle=\tiny\color{mygray}, % the style that is used for the line-numbers
rulecolor=\color{black}, % if not set, the frame-color may be changed on line-breaks within not-black text (e.g. comments (green here))
showspaces=false, % show spaces everywhere adding particular underscores; it overrides 'showstringspaces'
showstringspaces=false, % underline spaces within strings only
showtabs=false, % show tabs within strings adding particular underscores
stepnumber=1, % the step between two line-numbers. If it's 1, each line will be numbered
stringstyle=\color{mymauve}, % string literal style
tabsize=2, % sets default tabsize to 2 spaces
title=\lstname % show the filename of files included with \lstinputlisting; also try caption instead of title
}
%END of listing package%
 
\definecolor{darkgray}{rgb}{.4,.4,.4}
\definecolor{purple}{rgb}{0.35, 0.3, 0.82}
 
%define Javascript language
\lstdefinelanguage{JavaScript}{
keywords={typeof, new, true, false, catch, function, return, null, catch, switch, var, if, in, while, do, else, case, break},
keywordstyle=\color{blue}\bfseries,
ndkeywords={class, export, boolean, throw, implements, import, this},
ndkeywordstyle=\color{darkgray}\bfseries,
identifierstyle=\color{black},
sensitive=false,
comment=[l]{//},
morecomment=[s]{/*}{*/},
commentstyle=\color{purple}\ttfamily,
stringstyle=\color{red}\ttfamily,
morestring=[b]',
morestring=[b]"
}
 
\lstset{
language=JavaScript,
extendedchars=true,
basicstyle=\footnotesize\ttfamily,
showstringspaces=false,
showspaces=false,
numbers=left,
numberstyle=\footnotesize,
numbersep=9pt,
tabsize=2,
breaklines=true,
showtabs=false,
captionpos=b
}

\makeatother
\begin{document}
\begin{lstlisting}[language=JavaScript]
/******************** EXERCICIO 5 **************************/
/* a. */console.log(Math.sqrt((45*2-30/5-8),2.0)) // 8.717797887081348
/* b. */console.log(Math.pow(-3, 3)) // -27 
/* c. */console.log(27 - Math.pow(3, 4)) // -54 
/* d. */console.log(70 - (50/2)*5*3) // -305
/* e. */console.log(Math.round(Math.pow(-70.75, 4))) // 25055656 

/******************** EXERCICIO 6 **************************/
let A,B,C,D,E

A = 20 // => 20

B = (40 + A)/3 
/* => (20 + 20) -> 70 
 * 70/3 -> 20 */

C = Math.sqrt((A + 80), 2.0)
/* => 20 + 80 -> 100
 * => sqrt(100,2.0) -> 10
 * */

D = (A >= B) // => (20 >= 20) -> true

E = (C == B) // => (10 == 20) -> false

console.log(A, " ", " ", B, " ", C, " ", D, " ", E)


/******************** EXERCICIO 7 **************************/

/* a. */ console.log("mario" == "maria") // false
/* b. */ console.log(2 + 4 == 6) // true
/* c. */ console.log(10 - 4 > 7) // false
/* d. */ console.log((2*3)>(3*2)) // false
/* e. */ console.log(!('a' > 'A')) // ????

/* Resultado: false, true, false, false, false */


/******************** EXERCICIO 8 **************************/

function retornaMaiorNumero(n1, n2, n3) {
	return Math.max(n1, n2, n3)
}

const input1 = require('prompt-sync')({sigint:true})
const input2 = require('prompt-sync')({sigint:true})
const input3 = require('prompt-sync')({sigint:true})

const num1 = input1("Insira o primeiro numero ")
const num2 = input2("Insira o segundo numero ")
const num3 = input3("Insira o terceiro numero ")

console.log(num1, num2, num3)

console.log("O maior numero e ", retornaMaiorNumero(num1, num2, num3))



/******************** EXERCICIO 9 **************************/

let myArray = [3, 12, 4, 15, 1 , 2, 7, 8]

function imprimeDoisMaiores(array) { 
	var max = Math.max.apply(null, array) /* Encontrando o maior elemento */ 
	console.log("Maior elemento:", max)

	array.splice(array.indexOf(max), 1) /* Excluindo o maior elemento do array */ 

	console.log("Segundo maior elemento:", Math.max.apply(null, array)) /* Encontrando e imprimindo o proximo maior */ 
}

imprimeDoisMaiores(myArray)


/******************** EXERCICIO 10 **************************/

const ateFicarMaior = (altura1, taxaDeCrescimento1, altura2, taxaDeCrescimento2) => {
	let ano = 0
	for (ano = 0; altura2 <= altura1; ano++) {
		altura1 += taxaDeCrescimento1
		altura2 += taxaDeCrescimento2
	}
	return ano
}

let anacleto = {
	alturaCM: 150,
	taxaDeCrescimentoAno: 2 
}

let felisberto = {
	alturaCM: 110,
	taxaDeCrescimentoAno: 3 
}


let passou = ateFicarMaior(anacleto.alturaCM, anacleto.taxaDeCrescimentoAno, felisberto.alturaCM, felisberto.taxaDeCrescimentoAno)

console.log("Anacleto passara Felisberto apos " + passou + " anos, estando com 2m33cm e seu irmao com 2m32cm!")
\end{lstlisting}
\end{document}
